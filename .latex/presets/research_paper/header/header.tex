\documentclass{article}\makeatletter\usepackage{xparse}
%  ┌────────────────────────┐
%  │        Packages        │
%  └────────────────────────┘
    % necessary packages
        \usepackage{fontspec}   % allows Unicode in XeLaTex
        % \usepackage{ucharclasses} % allows Unicode in XeLaTex
        \usepackage[utf8]{inputenc}	% allows new character options
        \usepackage[a4paper]{geometry} % Paper dimensions and margins
        \usepackage{titletoc}   % change Table of Contents settings
        \usepackage{etoolbox}   % alters toc settings (learn in future)
        \usepackage{fancyhdr}   % include document header
        \usepackage{hyperref}	% creates hyper-link color options
        \usepackage{url}    	% package for url links
        \usepackage{graphicx}	% manage images and graphics
        \usepackage{subcaption} % add two plots together
        \usepackage{color}      % include colors for 
        \usepackage{floatrow}	% allows placement of figures [H]
        \usepackage{enumitem}   % create lists
        \usepackage{scrextend}  % \begin{addmargin}[<left indent>]{<indent>}
        \usepackage{listings, lstautogobble} % includes ability to input code
        \usepackage{cleveref}	% (\Cref) include "Figure" on \reff 

    % tables
        \usepackage{booktabs}   % allows for \toprule in tables
        % \usepackage{tabulary}	% table columns size of their contents (on page)
        \usepackage{tabularx}
        \newcolumntype{b}{>{\hsize=2.3\hsize}X}
        \newcolumntype{s}{>{\hsize=.45\hsize}X}
        \newcolumntype{m}{>{\hsize=.9\hsize}X}

    % checkmark with tikz
        % \usepackage{tikz}
        % \def\checkmark{\tikz\fill[scale=0.4](0,.35) -- (.25,0) -- (1,.7) -- (.25,.15) -- cycle;} 

    % math packages
        % \usepackage{bm}         % use of bold characters in math mode
        % \usepackage{amsmath}    % allows equations to be split
        % \usepackage[nomessages]{fp} % calculations in latex
        % % function to create magnitude bars around a function
        %     \newcommand\longvar[1]{\mathchardef\UrlBreakPenalty=100
        %     \mathchardef\UrlBigBreakPenalty=100\url{#1}}
        %     \newcommand{\norm}[1]{\left\lVert#1\right\rVert}


    % packages used in past
        % \usepackage{pdfpages}   % include entire pdf documents
        % \usepackage{xstring}    % StrSubstitute replace character
        \usepackage{caption}    % removes figure from LoF: \caption[]{}

    % code and file variables
    \definecolor{light-gray}{gray}{0.95}    % gray that stack exchange uses
    \definecolor{mygreen}{RGB}{28,172,0}	% custom defined colors
    \definecolor{mylilas}{RGB}{170,55,241}
    \definecolor{mymauve}{rgb}{0.58,0,0.82}

    \lstset {
        language=Python,
        backgroundcolor = \color{light-gray},
        breaklines		= true,
        tabsize         = 4,
        keywordstyle    = \color{blue},
        morekeywords    = [2]{1}, keywordstyle=[2]{\color{black}},
        identifierstyle = \color{black},
        stringstyle     = \color{mylilas},
        commentstyle    = \color{mygreen},
        numbers         = left,
        numberstyle     = {\tiny \color{black}},	% size of the numbers
        numbersep       = 6pt, 						% distance of numbers from text
        emph            = [1]{as, for, end, break}, % bold for, end, break...
        emphstyle 		= [1]\color{red}, 			% emphasis
        basicstyle		= \footnotesize\ttfamily,	% set font to courier
        frameround      = ffff,                     % TR, BR, BL, TL. t(round)|f(flat)
        frame           = single,                   % single line all around
        showstringspaces= false,                    % blank spaces appear as written
        autogobble      = true
    }

%  ┌────────────────────────┐
%  │      Common Tasks      │
%  └────────────────────────┘
    \DeclareDocumentCommand{\commontasks}{m} {
        \href{https://drive.google.com/open?id=0B5NW7S3txe5UTE0xSHJHNWxJbEE}{\underline{this link}} 
        \mono{PYTHON}
        \lstinputlisting[language=Python]{../python/3_linear.py}

        % two figures side by side
        \begin{figure}[H]
            \includegraphics[width=.45\linewidth]{./media/1-nearest.pdf}\hfill 
            \includegraphics[width=.45\linewidth]{./media/5-nearest.pdf}
            \caption[k=1 and k=5 Nearest Neighbor]{Left: k=1 Nearest Neighbor. Right: k=5 Nearest Neighbor}
            \label{fig:kNearOneAndFive}
        \end{figure}
    }

%  ┌────────────────────────┐
%  │   General Functions    │
%  └────────────────────────┘
    \DeclareDocumentCommand{\svg}{m m m m m m} { 
        % inkscape -D -z --file=network.svg --export-pdf=network.pdf --export-latex
        \begin{figure}[#2]
            \centering
            #3
            \def\svgwidth{#4\linewidth}
            \input{#1}
            \caption{#5}
            \label{#6}
        \end{figure}
    }

    \DeclareDocumentCommand{\tikz}{mmm} {
        \begin{figure}[H]
            \begin{center}
                \includegraphics[]{#1}
                \caption{#2}
                \label{#3} 
            \end{center}
        \end{figure} 
    }
    
    \DeclareDocumentCommand{\mono}{mm} {
        \sloppy
        \lstinline{#1} 
        \hspace{-.7em}
        \hspace{#2em}
    }

    \DeclareDocumentCommand{\matPlot}{mmmm} {
        \begin{figure}[H]
            \centering
            \renewcommand{\figurescale}{#2} 
            \input{#1}
            \caption{#3}
            \label{#4}
        \end{figure}
    }

    \DeclareDocumentCommand{\matPlotTwo}{mmmmmmm} {
        \begin{figure}[H]
        \centering
        \begin{subfigure}{.5\textwidth}
          \centering
            \renewcommand{\figurescale}{#2} 
            \input{#1}
          \caption{}
        \end{subfigure}%
        \begin{subfigure}{.5\textwidth}
          \centering
            \renewcommand{\figurescale}{#4} 
            \input{#3}
          \caption{}
        \end{subfigure}
        \caption[#6]{#5}
        \label{#7}
        \end{figure}
    }

    \DeclareDocumentCommand{\margin}{m} {
        \begin{addmargin}[4em]{0em}
            #1 
        \end{addmargin} 
    }

    \DeclareDocumentCommand{\newFigureTwo}{mmmmmmm} {
        % \FPeval{opposite}{1-#9}
        \begin{figure}[H]
        \centering
        \begin{subfigure}{.5\textwidth}
          \centering
          \includegraphics[width=#2\linewidth]{#1}
          \caption{}
        \end{subfigure}%
        \begin{subfigure}{.5\textwidth}
          \centering
          \includegraphics[width=#4\linewidth]{#3}
          \caption{}
        \end{subfigure}
        \caption[#6]{#5}
        \label{#7}
        \end{figure}
    }

    \DeclareDocumentCommand{\lstPy}{m m m o o} {
        % \lstinputlisting[language=C++,firstline=#4,lastline=#5,caption={#2},captionpos=b,label={#3}]{#1}
        \lstinputlisting[language=Python,caption={#2},label={#3},firstline=#4,lastline=#5]{#1}%
    }

    \DeclareDocumentCommand{\lstCpp}{m m m o o} {
        % \lstinputlisting[language=C++,firstline=#4,lastline=#5,caption={#2},captionpos=b,label={#3}]{#1}
        \lstinputlisting[language=C++,caption={#2},label={#3},firstline=#4,lastline=#5]{#1}%
    }

    \DeclareDocumentCommand{\lstMat}{m m m o o} {
        \lstinputlisting[language=Matlab,caption={#2},label={#3},firstline=#4,lastline=#5]{#1}%
    }

    \DeclareDocumentCommand{\reff}{m} {
        \edef\link{#1}
        \hspace{-0.5em}\hyperref[\link]{\Cref*{\link}} \hspace{-0.65em}
    }

    \DeclareDocumentCommand{\newFigure}{m m m m} {
        \edef\path{#1} \edef\figcaption{#2} \edef\size{#3}  
        \begin{figure}[H]
            \begin{center}
                \includegraphics[width=\size\textwidth]{\path}
                \caption{\figcaption}
                \label{#4} % label gets rid of type and -.
            \end{center}
        \end{figure} 
    }

%  ┌────────────────────────┐
%  │     Tikz Settings      │
%  └────────────────────────┘
    % % tikz packages
    % \usepackage{tikz}
    % \usepackage{tikzscale}
    % \usetikzlibrary{plotmarks}
    % \usetikzlibrary{arrows.meta}
    % \usetikzlibrary{shapes.geometric, arrows}
    % \usepackage{grffile}
    % \usepackage{amsmath}
    % \usepackage{pgfplots} 
    % \usepackage{pgfgantt}
    % \usepackage{pdflscape}
    % \pgfplotsset{compat=newest} 
    % \pgfplotsset{plot coordinates/math parser=false}
    % \newlength\figureheight 
    % \newlength\figurewidth 
    % \newcommand\figurescale{1} % set default scaling to 1
    %
    % \tikzstyle{sampler} = [rectangle, rounded corners, 
    %                        minimum width=1cm, minimum height=1cm,
    %                        text centered, draw=black]
    % \tikzstyle{filter}  = [rectangle, minimum width=2cm, 
    %                        minimum height=1cm, text centered, 
    %                        draw=black] %fill=orange!30
    % \tikzstyle{box}     = [rectangle, minimum width=1cm, 
    %                        minimum height=1cm, text centered, 
    %                        draw=black] %fill=orange!30
    % \tikzstyle{arrow}   = [thick,->,>=stealth]
    % \tikzstyle{sum}     = [draw, fill=white, circle,
    %                        minimum height=0.6cm]
    %
    % \begin{filecontents*}{sampler.tikz}
    %     \begin{tikzpicture}[node distance=3cm]
    %         \node (x)   {$x[n]$};
    %         \node (u3)  [sampler, right of=x] {$↑$ 3};
    %         \node (h)   [filter, right of=u3] {$H(z)$};
    %         \node (d2)  [sampler, right of=h] {$↓$ 2};
    %         \node (y)   [right of=d2] {$y[n]$};
    %
    %         \draw [arrow] (x)   -> (u3);
    %         \draw [arrow] (u3)  -> (h);
    %         \draw [arrow] (h)node[anchor=south,yshift=0.5cm]{Low-pass Filter} -> (d2);
    %         \draw [arrow] (d2)  -> (y);
    %     \end{tikzpicture}
    % \end{filecontents*}

% \RenewDocumentCommand{\matPlot}{mmmm}{}
% \RenewDocumentCommand{\newFigure}{mmmm}{}
% \RenewDocumentCommand{\matPlotTwo}{mmmmmmm}{}
% \RenewDocumentCommand{\tikz}{mmm}{}
